\documentclass{article}


\usepackage{arxiv}

\usepackage[utf8]{inputenc} % allow utf-8 input
\usepackage[T1]{fontenc}    % use 8-bit T1 fonts
\usepackage{hyperref}       % hyperlinks
\usepackage{url}            % simple URL typesetting
\usepackage{booktabs}       % professional-quality tables
\usepackage{amsfonts}       % blackboard math symbols
\usepackage{nicefrac}       % compact symbols for 1/2, etc.
\usepackage{microtype}      % microtypography
\usepackage{lipsum}
\usepackage{amsmath}

\title{Use MCMC to do story generation}

\author{
  Shan Zhong\thanks{Shan is David's student)---\emph{not} for acknowledging funding agencies.} \\
  Department of Statistics\\
  University of Douth Carolina\\
  Columbia, SC 29208\\
  \texttt{zhongs@email.sc.edu} \\
  %% \AND
  %% Coauthor \\
  %% Affiliation \\
  %% Address \\
  %% \texttt{email} \\
  %% \And
  %% Coauthor \\
  %% Affiliation \\
  %% Address \\
  %% \texttt{email} \\
  %% \And
  %% Coauthor \\
  %% Affiliation \\
  %% Address \\
  %% \texttt{email} \\
}

\begin{document}
\maketitle

\begin{abstract}
This is a review of the papaer Toward Automated Story Generation with Markov Chain Monte Carlo Methods and Deep Neural Networks by Brent Harrison, Christopher Purdy, Mark O. Riedl. This paper tried to use both MCMC and Recurrent Neural Network to generate novel stories. I will apply both of the method to my own insurance data and check the performance. Furthermore, to see if I can modify and generated something on my own.
\end{abstract}


% keywords can be removed
\keywords{Recurrent Neural Network \and Bayesian  \and MCMC}


\section{Introduction}
How do we represent meaning of words? One solution is to use WordNet(a lexical database for the English language), WordNet use human labors to find synoym sets.  Then we can know "good" is similar to "nice".  But in this way, we cannot calculate a numerical similarity for those words. 
$$$$
The other way is to use word vectors. One way to do so is use one hot encoding to encode words as vectors, then the length of the vector is the size of the vocabulary. For example, the word "you" , "me" and "him" can then be represented as

$$
you =
\begin{pmatrix}
1 \\
0 \\
0 \\
0\\
\vdots \\
0 \\
\end{pmatrix}
me=
\begin{pmatrix}
0 \\
1\\
0 \\
0\\
\vdots \\
0 \\
\end{pmatrix}
him=
\begin{pmatrix}
0 \\
0\\
1 \\
0\\
\vdots \\
0 \\
\end{pmatrix}
$$

A better way to do so is to represent words in a fixed dimension. For example, we can have a dimension to represent tense , a dimension to represent plural, 
and another dimension to represent negate prefix, etc....  We then let machine to learn the vectors itself.
$$$$
How does the prediction works? There is the fancy idea "A word’s meaning is given by the words that frequently appear close-by" ,we can build a model to predict occurrence of a word given another word is next to it. We can then use the posterior distribution to sample tokens and generate stories.

One way is to use nerual networks, the other is use Markov Chain

\section{Bigram and MCMC}
In a bigram setting, the aurthor assume the distribution of tokens is conditional on the word previous to it. 
 \cite{Event representations for automated story generation}



Markov Chain Monte Carlo Simulation
\label{sec:headings}

\lipsum[4] See Section \ref{sec:headings}.

\subsection{Headings: second level}
\lipsum[5]
\begin{equation}
\xi _{ij}(t)=P(x_{t}=i,x_{t+1}=j|y,v,w;\theta)= {\frac {\alpha _{i}(t)a^{w_t}_{ij}\beta _{j}(t+1)b^{v_{t+1}}_{j}(y_{t+1})}{\sum _{i=1}^{N} \sum _{j=1}^{N} \alpha _{i}(t)a^{w_t}_{ij}\beta _{j}(t+1)b^{v_{t+1}}_{j}(y_{t+1})}}
\end{equation}

\subsubsection{Headings: third level}
\lipsum[6]

\paragraph{Paragraph}
\lipsum[7]

\section{Examples of citations, figures, tables, references}
\label{sec:others}
\lipsum[8] \cite{kour2014real,kour2014fast} and see \cite{hadash2018estimate}.

The documentation for \verb+natbib+ may be found at
\begin{center}
  \url{http://mirrors.ctan.org/macros/latex/contrib/natbib/natnotes.pdf}
\end{center}
Of note is the command \verb+\citet+, which produces citations
appropriate for use in inline text.  For example,
\begin{verbatim}
   \citet{hasselmo} investigated\dots
\end{verbatim}
produces
\begin{quote}
  Hasselmo, et al.\ (1995) investigated\dots
\end{quote}

\begin{center}
  \url{https://www.ctan.org/pkg/booktabs}
\end{center}


\subsection{Figures}
\lipsum[10] 
See Figure \ref{fig:fig1}. Here is how you add footnotes. \footnote{Sample of the first footnote.}
\lipsum[11] 

\begin{figure}
  \centering
  \fbox{\rule[-.5cm]{4cm}{4cm} \rule[-.5cm]{4cm}{0cm}}
  \caption{Sample figure caption.}
  \label{fig:fig1}
\end{figure}

\subsection{Tables}
\lipsum[12]
See awesome Table~\ref{tab:table}.

\begin{table}
 \caption{Sample table title}
  \centering
  \begin{tabular}{lll}
    \toprule
    \multicolumn{2}{c}{Part}                   \\
    \cmidrule(r){1-2}
    Name     & Description     & Size ($\mu$m) \\
    \midrule
    Dendrite & Input terminal  & $\sim$100     \\
    Axon     & Output terminal & $\sim$10      \\
    Soma     & Cell body       & up to $10^6$  \\
    \bottomrule
  \end{tabular}
  \label{tab:table}
\end{table}

\subsection{Lists}
\begin{itemize}
\item Lorem ipsum dolor sit amet
\item consectetur adipiscing elit. 
\item Aliquam dignissim blandit est, in dictum tortor gravida eget. In ac rutrum magna.
\end{itemize}


\bibliographystyle{unsrt}  
%\bibliography{references}  %%% Remove comment to use the external .bib file (using bibtex).
%%% and comment out the ``thebibliography'' section.


%%% Comment out this section when you \bibliography{references} is enabled.
\begin{thebibliography}{1}

\bibitem{Event representations for automated story generation}
Martin, L. J.; Ammanabrolu, P.; Hancock, W.; Singh, S.; Harrison, B.; and Riedl, M. O. 
\newblock Event representations for automated story generation with deep neural nets
\newblock In arXiv:1706.01331.

\bibitem{kour2014fast}
George Kour and Raid Saabne.
\newblock Fast classification of handwritten on-line arabic characters.
\newblock In {\em Soft Computing and Pattern Recognition (SoCPaR), 2014 6th
  International Conference of}, pages 312--318. IEEE, 2014.

\bibitem{hadash2018estimate}
Guy Hadash, Einat Kermany, Boaz Carmeli, Ofer Lavi, George Kour, and Alon
  Jacovi.
\newblock Estimate and replace: A novel approach to integrating deep neural
  networks with existing applications.
\newblock {\em arXiv preprint arXiv:1804.09028}, 2018.

\end{thebibliography}


\end{document}
